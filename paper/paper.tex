\documentclass[a4paper,12pt]{article}

\title{\textbf{FeynmanJS}: Drawing Portable Feynman Diagrams with JavaScript and Scalable Vector Graphics (SVG)}
\date{May 2014}
\author{Marcell Jusztin}

\begin{document}
\maketitle
    
\begin{abstract}
FeynmanJS is a JavaScript tool for producing high quality, Scalable Vector Graphic (SVG) Feynman diagrams for the modern web. It is based on the popular \textit{feynMF}\cite{fmfmanual} package for \LaTeX, featuring automatic layout management and support for multiple syntaxes.

\end{abstract}

\section{Introduction}

\subsection{Purpose and Scope}
In the past few years the internet has gone through a drastic evolution changing the way we share information with each other. With online journals and  encyclopedias such as Wikipedia becoming available to a broader audience one might find need for a quicker and easier way of sharing graphically rich, scientific -- or other -- content\cite{zanpan}.

\subsection{Overview}


\begin{thebibliography}{10}

\bibitem{fmfmanual}
  Thorsten Ohl,
  \emph{feynMF: Drawing Feynman Diagrams with \LaTeX and METAFONT}.
  Technische Hochschule Darmstadt, Darmstadt, Germany
  1997.

\bibitem{zanpan}
  Zan Pan,
  \emph{jQuery.Feyn: Drawing Feynman Diagrams with SVG}.
  Institute of Theoretical Physics,
  Chinese Academy of Sciences, Beijing, China
  2013.

\end{thebibliography}

\end{document}